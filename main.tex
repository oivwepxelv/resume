%%%%%%%%%%%%%%%%%
% This resume was created using a version of altacv.cls (v1.1.4, 27 July 2018)
% originally written by LianTze Lim (liantze@gmail.com) and modified by me, based
% a CV created by BusinessInsider.
%%%%%%%%%%%%%%%%

% Preamble
\documentclass[8pt,a4paper]{altacv}

% Packages
\usepackage{enumitem}
\usepackage[T1]{fontenc}
\usepackage[hidelinks]{hyperref}
\usepackage[default]{lato} % For pdflatex, \setmainfont{Lato} for xelatex or lualatex
\usepackage{xcolor}

% Page layout
\geometry{
    left=0.75in,
    right=3.45in,
    marginparwidth=2.05in,
    marginparsep=0.25in,
    top=0.75in,
    bottom=0.75in
}

% Set colors
\definecolor{Orange}{HTML}{FF7F00}
\definecolor{SlateGrey}{HTML}{2E2E2E}
\definecolor{LightGrey}{HTML}{666666}
\colorlet{heading}{Orange}
\colorlet{accent}{Orange}
\colorlet{emphasis}{SlateGrey}
\colorlet{body}{LightGrey}

% Set bullets for itemize and rating marker
% \renewcommand{\itemmarker}{\raisebox{-2pt}{\color{accent}\textbullet}}
\renewcommand{\ratingmarker}{\faCircle}

% Set hyperlink formatting
% \hypersetup{
%     colorlinks=false,
%     pdfborderstyle={/S/U/W 0.1},
%     linkbordercolor=body!30,
%     urlbordercolor=body!30,
%     citebordercolor=body!30,
% }


\begin{document}
    \name{Alex DeMaio}
    % \tagline{Senior Data Engineer}
    \personalinfo{
        \email{\href{mailto:oivwepxelv@gmail.com}{oivwepxelv@gmail.com}}
        \linkedin{\href{https://www.linkedin.com/in/oivwepxelv}{linkedin.com/in/oivwepxelv}}
        \github{\href{https://www.github.com/oivwepxelv}{github.com/oivwepxelv}}
        \phone{\href{tel:+17327591900}{732 759-1900}}
        % \location{NYC Metropolitan Area}
    }
    
    % Extend header to full width rather than half width
    \begin{fullwidth}
        \makecvheader
    \end{fullwidth}    
    
    % Set itemize font
    \AtBeginEnvironment{itemize}{\small}
    
    %%%%%%%%%%%%%%%%
    % Begin Relevant Work Experience
    %%%%%%%%%%%%%%%%
    \cvsection[rightsidebar]{Work Experience}
        \cvevent{Spaulding Ridge, LLC (formerly Data Clymer, LLC)}
                {Senior Data Engineer}
                {2024 - Present}
                {Remote}
                {}
        \vspace{-8pt}
        \cvevent{}
                {Data Engineer}
                {2022 - 2024}
                {Remote}
                {\begin{itemize}[leftmargin=0pt]
                \renewcommand\labelitemi{}
                    \item Created and maintained data pipelines end-to-end for clients in multiple industries (from ingestion to warehousing, modeling, visualization, and analytics)
                    \vspace{1.5pt}
                    \item {\bfseries\color{emphasis}Projects:}
                        \begin{itemize}[leftmargin=2.25em]
                        \smallskip
                        \item[\raisebox{-2pt}{\color{accent}\Large\faCode}] Worked with software team on DevOps / Python backend development of web app; wrote ingestion pipelines for third-party data sources to Google BigQuery and PostgreSQL databases; created GraphQL endpoints to provide data access for frontend development 
                        \smallskip
                        \item[\raisebox{-2pt}{\color{accent}\Large\faCalendar}] Maintained and augmented an automated event-management utility, using Selenium scraping of event-hosting platforms (and APIs where available) to consolidate calendar updates, guest contact lists, and event metrics for market research
                        \smallskip
                        \item[\raisebox{-2pt}{\color{accent}\Large\faBarChart}] Used dbt, Fivetran, GCP, Looker, and Python to maintain and augment cloud-based CRM and financial data pipelines for a venture capital firm; expanded data quality monitoring / alerting and security protocol
                        \smallskip
                        \item[\raisebox{-2pt}{\color{accent}\Large\faMedkit}] Used dbt and GCP to correct and reconstruct broken, inefficient R-based ETL pipeline for healthcare data
                    \end{itemize}
                \end{itemize}}
            
        \cvevent{Marmalade Labs}
                {Data Analyst}
                {2021-2022}
                {Remote}
                {\begin{itemize}[leftmargin=0pt]
                \renewcommand\labelitemi{}
                    \item Used dbt, GCP, and Python to construct dashboards for tracking website analytics and backend performance metrics, for use by backend / ML engineering team; A / B tested ads; oversaw 10 annotation contractors for training ML model powering search
                \end{itemize}}
            
    %%%%%%%%%%%%%%%%
    % Begin Education
    %%%%%%%%%%%%%%%%
    \cvsection{Education} % Provide tex file name "page1sidebar" containing the sidebar contents
    
        \cvevent{Stanford University}
                {Ph.D. student, Physics (left after earning M.S.)}
                {2015 - 2017}
                {Stanford, CA}
                {\begin{itemize}[leftmargin=0pt]
                \renewcommand\labelitemi{}
                    \item {\color{emphasis}Coursework:} General Relativity, E\&M, Quantum Field Theory, Quantum Mechanics, Statistical Mechanics
                \end{itemize}}
            
        \cvevent{Rutgers University (Honors Program)}
                {B.S. Mathematics (GPA 4.0 / 4.0)}
                {2011 - 2015}
                {New Brunswick, NJ}
                {}
        \vspace{-8pt} % Decrease spacing for events with same org
        \cvevent{} % Skip duplicative org
                {B.S. Physics (GPA 4.0 / 4.0, Highest Honors)}
                {2011 - 2015}
                {New Brunswick, NJ}
                {\begin{itemize}[leftmargin=0pt]
                \renewcommand\labelitemi{}
                    \item {\color{emphasis}Coursework:} Abstract Algebra / Group Theory, Complex Analysis, Differential Geometry, E\&M, Mechanics, ODE / PDE, Probability / Statistics, Particle Physics, Quantum Mechanics, Real Analysis
                \end{itemize}}
    
    %%%%%%%%%%%%%%%%
    % Research / Internships
    %%%%%%%%%%%%%%%%
    \cvsection{Research / Internships}
        \cvevent{Stanford University}
                {Research Assistant}
                {2015-2016}
                {Stanford, CA}
                {\begin{itemize}[leftmargin=0pt]
                \renewcommand\labelitemi{}
                    \item Quarterly rotations in different research areas / groups on campus (feature of the Ph.D. program for first-years)
                \end{itemize}}

    \cvevent{Rutgers University}
            {Research Assistant}
            {2013-2015 (Academic Term)}
            {New Brunswick, NJ}
            {}
    \vspace{-8pt}
    \cvevent{}
            {Research Fellow}
            {2012}
            {New Brunswick, NJ}
            {\begin{itemize}[leftmargin=0pt]
            \renewcommand\labelitemi{}
                \item As part of the Rutgers Heavy-Ion Group / CERN CMS collaboration, used LHC collision data and simulated collision data to answer questions related to the study of quark-gluon plasma (C++ / ROOT for data analysis) 
            \end{itemize}}
    
    \cvevent{CERN}
            {Research Fellow}
            {2014}
            {Geneva, Switzerland}
            {\begin{itemize}[leftmargin=0pt]
            \renewcommand\labelitemi{}
                \item {\color{emphasis}One of 15 students in the USA selected for NSF-funded travel, board, research, and coursework at the CERN facility in Geneva for the summer}
                \item Produced visualizations of the beam shape / dynamics in all of the different components of the LHC
            \end{itemize}}
    
    \cvevent{Caltech}
            {Research Fellow}
            {2013}
            {Pasadena, CA}
            {\begin{itemize}[leftmargin=0pt]
            \renewcommand\labelitemi{}
                \item {\color{emphasis}One of $\sim$30 students in the USA funded by Caltech to travel to campus and conduct gravitational-wave research for the summer}
                \item Our group (part of the Caltech Theoretical Astrophysics Group) wrote and ran simulations of core-collapse supernovae (CCSN), then inferred CCSN properties from the resulting gravitational waveform data using matched filtering analysis and Bayesian model selection
                \end{itemize}}

\end{document}
